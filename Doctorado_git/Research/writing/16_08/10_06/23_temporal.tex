\documentclass[12pt]{report}
\usepackage[utf8x]{inputenc}
\usepackage{amsmath}
\usepackage{amsfonts}
\usepackage{mathrsfs}
\usepackage{natbib}
\usepackage{graphicx} % figuras
%\usepackage[export]{adjustbox} % loads also graphicx
\usepackage{float}
\usepackage[font=footnotesize]{caption}
\usepackage{wrapfig}
%\usepackage{authblk}
%\usepackage{subfigure}
\usepackage{pifont}
\usepackage{a4wide}
\usepackage{wrapfig}
 \usepackage{tabu}
%\usepackage{multirow}% 

%\graphicspath{/home/wagm/cortes/Localdisk/Results/16_05/05_09/}
\topmargin=-2pt
\title{Physics-based pre-conditioners for large-scale subsurface flow simulation}

%\author[1]{G. B. Diaz Cortes}  
%\author[1]{C. Vuik} 
%\author[2]{J. D. Jansen} 
%\affil[1]{Department of Applied Mathematics, TU Delft}
%\affil[2]{Department of Geoscience \& Engineering, TU Delft}
%\renewcommand\Authands{ and }
%\date{March 2016}
\begin{document}
\thispagestyle{empty}
% \noindent
% \begin{center}
% {\Large \sc DELFT UNIVERSITY OF TECHNOLOGY}
% \\
% \vspace{3cm}
% {\large \sc REPORT 16-03}\\[4ex]
% {\large \sc Physics-based pre-conditioners for large-scale subsurface flow simulation}\\[4ex]
% {\large \sc G. B. Diaz Cortes, C. Vuik, J. D. Jansen}\\
% \vfill
% {\tt ISSN 1389-6520}\\[2ex]
% {\tt Reports of the Delft Institute of Applied Mathematics}\\[2ex]
% {\tt Delft 2016}
% \end{center}
% \pagebreak
% \thispagestyle{empty}
% \vspace*{\fill}
% \noindent
% \hspace*{-0,3cm}Copyright~~~\Pisymbol{psy}{227}~~~2016 by Delft Institute of Applied Mathematics, Delft, \mbox{The Netherlands.}
% \\[2ex]
% No part of the Journal may be reproduced, stored in a retrieval system, or
% transmitted, in any form or by any means, electronic, mechanical, photocopying,
% recording, or otherwise, without the prior written permission from Delft Institute of
% Applied Mathematics, Delft University of Technology, The
% Netherlands. 
% % newpage, title etc.
% \setcounter{page}{1}


% Title Page





I changed the stopping criterium. The new stopping criterium is:
$$\frac{||l^{-1}r||_2}{||l^{-1}b||_2}\leq tol+\frac{tolNR}{||x_k||_2},$$
with tolNR the tolerance of the NR method, and tol the tolerance of the linear solver.
\subsection*{Incompressible}
\begin{figure}[H]
 \centering
 \begin{minipage}{.5\textwidth}
\includegraphics[width=9cm,height=9cm,keepaspectratio]
%{/home/wagm/cortes/Localdisk/Results/16_06/21/size_35perm_0_5wells_5_defvect0c01/solution.jpg}
{/home/wagm/cortes/Localdisk/Results/16_06/23/perm_0snap_5_defvect3rNR_0_0/solution.jpg}
\caption{Solution, well fluxes}
\label{fig:insol1}
\end{minipage}%
\hspace{4mm}
\begin{minipage}{.45\textwidth}
 \centering
 \centering
\includegraphics[width=9cm,height=9cm,keepaspectratio]
{/home/wagm/cortes/Localdisk/Results/16_06/23/perm_0snap_5_defvect3rNR_0_0/iterations_4NR.jpg}
\caption{Number of iterations ICCG only}
\label{fig:initer1}
\end{minipage}%
\end{figure}%
\begin{figure}[H]
 \centering
 \begin{minipage}{.5\textwidth}
\includegraphics[width=9cm,height=9cm,keepaspectratio]
%{/home/wagm/cortes/Localdisk/Results/16_06/21/size_35perm_0_5wells_5_defvect0c01/solution.jpg}
{/home/wagm/cortes/Localdisk/Results/16_06/23/perm_0snap_5_defvect3rNR_0_1/solution.jpg}
\caption{Solution, well fluxes}
\label{fig:insol1}
\end{minipage}%
\hspace{4mm}
\begin{minipage}{.45\textwidth}
 \centering
 \centering
\includegraphics[width=9cm,height=9cm,keepaspectratio]
{/home/wagm/cortes/Localdisk/Results/16_06/23/perm_0snap_5_defvect3rNR_0_1/iterations_4NR.jpg}
\caption{Number of iterations ICCG and DICCG}
\label{fig:initer1}
\end{minipage}%
\end{figure}%



\subsection*{Compressible}
\begin{figure}[H]
 \centering
 \begin{minipage}{.5\textwidth}
\includegraphics[width=9cm,height=9cm,keepaspectratio]
%{/home/wagm/cortes/Localdisk/Results/16_06/21/size_35perm_0_5wells_5_defvect0c01/solution.jpg}
{/home/wagm/cortes/Localdisk/Results/16_06/23/perm_0snap_5_defvect3rNR_0_0c/solution.jpg}
\caption{Solution, well fluxes}
\label{fig:insol1}
\end{minipage}%
\hspace{4mm}
\begin{minipage}{.45\textwidth}
 \centering
 \centering
\includegraphics[width=9cm,height=9cm,keepaspectratio]
{/home/wagm/cortes/Localdisk/Results/16_06/23/perm_0snap_5_defvect3rNR_0_0c/iterations_4NR.jpg}
\caption{Number of iterations ICCG only}
\label{fig:initer1}
\end{minipage}%
\end{figure}%
\begin{figure}[H]
 \centering
 \begin{minipage}{.5\textwidth}
\includegraphics[width=9cm,height=9cm,keepaspectratio]
%{/home/wagm/cortes/Localdisk/Results/16_06/21/size_35perm_0_5wells_5_defvect0c01/solution.jpg}
{/home/wagm/cortes/Localdisk/Results/16_06/23/perm_0snap_5_defvect3rNR_0_1c/solution.jpg}
\caption{Solution, well fluxes}
\label{fig:insol1}
\end{minipage}%
\hspace{4mm}
\begin{minipage}{.45\textwidth}
 \centering
 \centering
\includegraphics[width=9cm,height=9cm,keepaspectratio]
{/home/wagm/cortes/Localdisk/Results/16_06/23/perm_0snap_5_defvect3rNR_0_1c/iterations_4NR.jpg}
\caption{Number of iterations ICCG and DICCG}
\label{fig:initer1}
\end{minipage}%
\end{figure}%

For the incompressible problem it works well, but for the compressible problem again we only observe changes
after the 2nd NR iteration with the deflation method.\\
The residual of the NR iteration is 
$$resNorm=norm(res)\leq tolNR,$$
and we use this residual as right hand side 
$$A=J, \qquad b=res, \qquad y=upd.$$
I was thinking in two details, the first one is that if we have a good solution for the previous time step, 
this means that the residual of the NR iteration is small, and this is our right hand side, then we are 
dividing by a small number and maybe this can cause some problems. \\
We want to solve $Ay=b$, so if we already have an accurate $||b||_2\approx tol$ then we should have 
$||Ay||_2\approx tol$. 
Then I was thinking that maybe we can use this in the stopping criterium for the linear solver. 




\end{document}